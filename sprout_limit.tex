\section{Limitations and Future Work}

Although our results are encouraging, there are several limitations to
our work. First, as noted in \S\ref{s:problem} and \S\ref{s:sprout},
an end-to-end system like Sprout cannot control delays when the
bottleneck link includes competing traffic that shares the same
queue. If a device uses traditional TCP outside of Sprout, the
incurred queueing delay---seen by Sprout and every flow---will be
substantial.

Sprout is not a traditional congestion-control protocol, in that it is
designed to adapt to varying link conditions, not varying cross
traffic. In a cellular link where users have their own queues on the
base station, interactive performance will likely be best when the
user runs bulk and interactive traffic \emph{inside} Sprout
(e.g.~using SproutTunnel), not alongside Sprout. We have not evaluated
the performance of multiple Sprouts sharing a queue.
% We leave to
%future work a better understanding of the {\em need} for active queue
%management to achieve good throughput-delay trade-offs with multiple
%independent contending flows, speculating here that a purely
%end-to-end approach may in fact work.

The accuracy of Sprout's forecasts depends on whether the application
is providing offered load sufficient to saturate the link. For
applications that switch intermittently on and off, or don't desire
high throughput, the transient behavior of Sprout's forecasts
(e.g.~ramp-up time) becomes more important. We did not evaluate any
non-saturating applications in this paper or attempt to measure or optimize
Sprout's startup time from idle.

Finally, we have tested Sprout only in trace-based emulation of eight
cellular links recorded in the Boston area in 2012. Although Sprout's
model was frozen before data were collected and was not ``tuned'' in
response to any particular network, we cannot know how generalizable
Sprout's algorithm is without more real-world testing.

In future work, we are eager to explore different stochastic network
models, including ones trained on empirical variations in cellular
link speed, to see whether it is possible to perform much better than
Sprout if a protocol has more accurate forecasts. We think it will be
worthwhile to collect enough traces to compile a standardized
benchmark of cellular link behavior, over which one could evaluate any
new transport protocol.
