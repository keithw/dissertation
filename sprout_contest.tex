\section{Sprout in context: the protocol-design contest}
\label{s:cc-contest}

In March and April of 2013, we ran a contest in MIT's graduate course in computer
networks to develop the ``best'' control protocol for a single long flow
running over a cellular network. Over a two-week period, we asked
students to develop algorithms that maximized the throughput divided
by the delay, also known as the ``power'' of a congestion-control
algorithm.

Forty students participated in the contest. At the outset, we gave
students a ``training'' trace from the Verizon LTE network and
Sprout's source code, and set up a dynamic leaderboard that showed the
best performance of every other student thus far, as well as results
from Skype and Sprout.\footnote{We also provided students with a
  second training trace to evaluate the robustness of their candidate
  algorithms.} During the course of the contest, the students made
almost 3,000 attempts to maximize power over the Verizon trace.

At the end of the two-week period, we froze the students' final
submitted algorithms and collected a fresh trace with the
Saturator, again while driving in greater Boston. The final results of the
contest were determined by running each of the submissions, and
Sprout, over the newly-collected trace. Teams that performed well on
this final evaluation were rewarded with gift certificates, and two
top-performing teams became co-authors on a forthcoming journal
publication describing the contest results and their
winning algorithms~\cite{cc-contest}.

\begin{figure}
\caption{Sprout, Skype, and 2,994 student-generated protocols running
  over an emulated Verizon LTE network trace. Sprout is near the
  efficient frontier determined empirically by the 40 students who
  participated in the contest. ``Omniscient'' represents the best
  achievable performance given foreknowledge of the times when packets
  will be delivered from the bottleneck queue.}
\vspace{\baselineskip}
\def\svgwidth{\columnwidth}\input{pointplot-withhull.pdf_tex}
\label{fig:convex-hull}
\end{figure}

Figure~\ref{fig:convex-hull} shows the performance of all of the
nearly 3,000 student-generated algorithms on the training trace. There
can be no assurance that the algorithms actually map out the range of
achievable throughput-delay tradeoffs. Our only rigorous outer bound
is the point labeled ``Omniscient,'' which represents the best
possible performance given foreknowledge of the link, so that packets
arrive at the queue exactly when the link is able to dequeue and
deliver the packet.

Nonetheless, if we treat the student submissions---trained on this
trace---as a proxy for the realizable range of throughput-delay
tradeoffs, Sprout was near the efficient frontier, despite having been
frozen before the trace was collected.

The results suggest that there may be an unavoidable tradeoff between
throughput and delay on this problem, and therefore no ``one size fits
all'' solution. We explore this notion further in
Chapter~\ref{chap:remy}.
