Sprout is an end-to-end transport protocol for interactive
applications that desire high throughput and low delay. Sprout works
well over cellular wireless networks, where link speeds change
dramatically with time, and current protocols build up multi-second
queues in network gateways. Sprout does not use TCP-style reactive
congestion control; instead the receiver observes the packet arrival
times to infer the uncertain dynamics of the network path. This
inference is used to forecast how many bytes may be sent by the
sender, while bounding the risk that packets will be delayed inside
the network for too long.

In evaluations on traces from four commercial LTE and 3G networks,
Sprout, compared with Skype, reduced self-inflicted end-to-end delay
by a factor of 7.9 and achieved 2.2$\times$ the transmitted bit rate
on average. Compared with Google's Hangout, Sprout reduced delay by a
factor of 7.2 while achieving 4.4$\times$ the bit rate, and compared
with Apple's Facetime, Sprout reduced delay by a factor of 8.7 with
1.9$\times$ the bit rate.

Although it is end-to-end, Sprout matched or outperformed TCP Cubic
running over the CoDel active queue management algorithm, which
requires changes to cellular carrier equipment to deploy. We also
tested Sprout as a tunnel to carry competing interactive and bulk
traffic (Skype and TCP Cubic), and found that Sprout was able to isolate
client application flows from one another.
